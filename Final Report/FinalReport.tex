\documentclass{article}
\usepackage{amsmath}
\usepackage{amssymb}
\usepackage[margin=1in]{geometry}
\usepackage{booktabs}
\usepackage{float}
\usepackage{graphicx}
\usepackage{hyperref}
\usepackage{pdfpages}

\title{Final Project\\Numerical Solutions to PDEs}
\date{9 May 2025}
\author{Connor Emmons and Noah Wells}

\begin{document}
\pagenumbering{gobble}
\maketitle
\begin{figure}[h]
    \centering
    \includegraphics[width=0.8\textwidth]{title.jpg}
\end{figure}
\vfill
\paragraph*{Documentation}
ChatGPT: \url{https://chatgpt.com/share/681eaa76-a3d8-800f-b691-38bc518f0b52}
GitHub: \url{https://github.com/Connor-Lemons/PDE-Final}
\newpage
\pagenumbering{arabic}

\section{Basic Heat Equation}

\subsection*{$u_t = 2u_{xx}\\u(0,t) = u(3,t) = 0\\u(x,0) = \sin(\pi x) + \sin(2\pi x)$}

First we must find the analytic solution to this equation. This is an extremely basic heat equation with neumann boundary conditions. The solution boils down to 
\[
\sum_{n=1}^{\infty}e^{-2\left(\frac{n\pi}{3}\right)^{2}t}\sin\left(\frac{n\pi x}{3}\right)
\]

which when given the initial conditions we see is equal to 

\[
e^{-2\pi^{2}t}\sin\left(\pi x\right)+e^{-8\pi^{2}t}\sin(2\pi x).
\]

Using the central difference approximation for second derivatives gives:
\begin{equation}
    \begin{aligned}
        \frac{du}{dt} &= 2\frac{u_{i+1}^n - 2u_i^n + u_{i-1}^n}{\Delta x^2}\\
        u(t_0) &= \sin(\pi x) + \sin(2\pi x)
    \end{aligned}
\end{equation}
Note that the index $i$ relates to the spatial steps and the index $n$ relates to the time steps. The standard RK4 formulation is given as:
\begin{equation}
    \begin{aligned}
        k_1 &= f(t_n, y_n) \\
        k_2 &= f\left(t_n + \frac{h}{2},\, y_n + \frac{h}{2}k_1\right) \\
        k_3 &= f\left(t_n + \frac{h}{2},\, y_n + \frac{h}{2}k_2\right) \\
        k_4 &= f(t_n + h,\, y_n + hk_3) \\
        y_{n+1} &= y_n + \frac{h}{6}\left(k_1 + 2k_2 + 2k_3 + k_4\right)
    \end{aligned}
\end{equation}
Adapting this to the problem at hand gives:
\begin{equation}
    \begin{aligned}
        k_1 &= \alpha \frac{u_{i+1}^{n-1} - 2u_i^{n-1} + u_{i-1}^{n-1}}{\Delta x^2} \\
        k_2 &= \alpha \frac{
            \left(u_{i+1}^{n-1} + \frac{\Delta t}{2}k_1\right)
            - 2\left(u_i^{n-1} + \frac{\Delta t}{2}k_1\right)
            + \left(u_{i-1}^{n-1} + \frac{\Delta t}{2}k_1\right)
        }{\Delta x^2} \\
        k_3 &= \alpha \frac{
            \left(u_{i+1}^{n-1} + \frac{\Delta t}{2}k_2\right)
            - 2\left(u_i^{n-1} + \frac{\Delta t}{2}k_2\right)
            + \left(u_{i-1}^{n-1} + \frac{\Delta t}{2}k_2\right)
        }{\Delta x^2} \\
        k_4 &= \alpha \frac{
            \left(u_{i+1}^{n-1} + \Delta t\, k_3\right)
            - 2\left(u_i^{n-1} + \Delta t\, k_3\right)
            + \left(u_{i-1}^{n-1} + \Delta t\, k_3\right)
        }{\Delta x^2} \\
        u_i^{n} &= u_i^{n-1} + \frac{\Delta t}{6} \left( k_1 + 2k_2 + 2k_3 + k_4 \right)
    \end{aligned}
\end{equation}
An example of the numerical solution plotted against the analytical solution is shown below. In order to view the full animation, please run the main.m file and select the corresponding option.
\begin{figure}[H]
    \centering
    \includegraphics[width=0.8\textwidth]{heat_example.jpg}
    \caption{Iteration 59 of RK4 Solution to Heat Equation}
\end{figure}
With the proper function definition in MatLab, it is easy to use ode45() to obtain an equivalent solution. Note that for this comparison to work directly, the full vector of $t$ values used by RK4 was passed to ode45() as an input. This means that while ode45() is still free to adjust the step size as necessary, the outputted solution will be on the specified values. Shown below is an example plot of all three solutions on top of each other. To view the full solution animated, please run the main.m file and select the corresponding option.
\begin{figure}[H]
    \centering
    \includegraphics[width=0.8\textwidth]{heat_all_example.jpg}
    \caption{Iteration 34 of RK4 Solution and ode45() Solution vs. Analytical Solution to Heat Equation}
\end{figure}

We found the maximum absolute error to be equal to 0.2178 and the error in the L2 norm to be equal to 16.0653. You can see the error of the numerical solution in the graph below. 

\begin{figure}[H]
    \centering
    \includegraphics[width=0.6\textwidth]{Error.png}
    \caption{Error of Numerical solution}
\end{figure}


\section{General Wave Equation}
\subsection*{$u_{tt} = 0.16u_{xx} + 0.02\sin(x+t)\\u(0,t) = 0.01\sin(t)\\u_x(2,t) = 0\\u(x,0) = \sin\left(\frac{\pi x}{2}\right)\\u_t(x,0) = \cos\left(\frac{\pi x}{2}\right)$}
Using the central difference approximation for second derivatives gives the following:
\begin{equation}
    \begin{aligned}
        &\frac{u_i^{n+1} - 2u_i^n + u_i^{n-1}}{\Delta t^2} = 0.16\left(\frac{u_{i+1}^n - 2u_i^n + u_{i-1}^n}{\Delta x^2}\right) + 0.02\sin\left(x+t\right)\\
        &u_i^{n+1} = \frac{0.16\Delta t^2}{\Delta x^2}\left(u_{i+1}^n - 2u_i^n + u_{i-1}^n\right) + 0.02\Delta t^2\sin\left(x+t\right) + 2u_i^n - u_i^{n-1}
    \end{aligned}
\end{equation}
Note that the $i$ index represents the spatial steps and the $n$ index represents the temporal steps. This gives a formulation for the next time step. Rewriting this gives a formulation for the current time step based on the previous time steps.
\begin{equation}
    u_i^{n} = \frac{0.16\Delta t^2}{\Delta x^2}\left(u_{i+1}^{n-1} - 2u_i^{n-1} + u_{i-1}^{n-1}\right) + 0.02\Delta t^2\sin\left(x+t\right) + 2u_i^{n-1} - u_i^{n-2}
\end{equation}
This is the form which will be implemented to produce the numerical solution. Note that this equation requires knowledge of the two previous time steps, and thus a different method is required for initialization. In order to do this, begin with the forward difference equation in time and the given initial condition.
\begin{equation}
    \frac{u_i^{n+1} - u_i^n}{\Delta t} = \cos\left(\frac{\pi x}{2}\right)
\end{equation}
For the wave equation, the other initial condition gives the value, but the wave equation requires two previous time steps. Using this formulation, a "ghost step" can be obtained and used to generate the first iteration of the wave equation. This is given by:
\begin{equation}
    u_i^0 = u_i^1 - \Delta t\cos\left(\frac{\pi x}{2}\right)
\end{equation}
For the boundary condition at the left endpoint, it is readily apparent that the formulation for this must be:
\begin{equation}
    u_{end}^n = u_{end-1}^n
\end{equation}
Shown below is an image of the wave at one of the iterations. To see the full animation, please run the main.m file and select the corresponding option.
\begin{figure}[H]
    \centering
    \includegraphics[width=0.8\textwidth]{wave_example.jpg}
    \caption{Iteration 2589 of Numerical Solution of Wave Equation}
\end{figure}

\section{Two-Dimensional General Heat Equation}
\subsection*{$u_t = 0.5\nabla^2u + e^{-t}u\\u_x(0,y,t) = 0\\u_y(x,0,t) = 0\\u(x,4,t) = e^{-t}\\u(4,y,t) = 0\\u(x,y,0) = \begin{cases}
1 & \text{for } y \geq x\\
0 & \text{for } y < x
\end{cases}$}
Using the forward difference approximation for first derivatives and the central difference approximation for second derivatives gives the following:
\begin{equation}
    \begin{aligned}
        &\frac{u_{i,j}^{n+1} - u_{i,j}^n}{\Delta t} = 0.5\left(\frac{u_{i+1,j}^n - 2u_{i,j}^n + u_{i-1,j}^n}{\Delta x^2} + \frac{u_{i,j+1}^n - 2u_{i,j}^n + u_{i,j-1}^n}{\Delta y^2}\right) + e^{-t}u_{i,j}^n\\
        &u_{i,j}^{n+1} = 0.5\Delta t\left(\frac{u_{i+1,j}^n - 2u_{i,j}^n + u_{i-1,j}^n}{\Delta x^2} + \frac{u_{i,j+1}^n - 2u_{i,j}^n + u_{i,j-1}^n}{\Delta y^2}\right) + \left(e^{-t}\Delta t + 1\right)u_{i,j}^n
    \end{aligned}
\end{equation}
In this case, $i$ and $j$ represent the $x$ and $y$ spatial steps, respectively. This gives a formulation for the next time step. Assuming the same step size for both spatial dimensions allows the simplificaiton $\Delta x = \Delta y = h$. Along with rewriting the formulation to give the current time step as a function of the previous time steps, this gives:
\begin{equation}
    u_{i,j}^n = \frac{0.5\Delta t}{h^2}\left(u_{i+1,j}^{n-1} + u_{i-1,j}^{n-1} + u_{i,j+1}^{n-1} + u_{i,j-1}^{n-1} - 4u_{i,j}^{n-1}\right) + \left(e^{-t}\Delta t + 1\right)u_{i,j}^{n-1}
\end{equation}
This is the form which will be implemented to produce the numerical solution. Similar to the wave equation above, for the derivative boundary conditions, it is readily apparent that the formulation for this must be:
\begin{equation}
    \begin{aligned}
        u_{start,j}^n = u_{start+1,j}^n\\
        u_{i,start}^n = u_{i,start+1}^n
    \end{aligned}
\end{equation}
Shown below is an image of the heat on the surface at one of the iterations. To see the full animation, please run the main.m file and select the corresponding option.
\begin{figure}[H]
    \centering
    \includegraphics[width=0.8\textwidth]{2D_example.jpg}
    \caption{Iteration 314 of Numerical Solution of 2D Heat Equation}
\end{figure}

\section{Poisson's Equation}
\subsection*{$\nabla^2u = 1 + 0.2\delta_{(1,3)}(x,y) = f(x,y)\\u(x,4) = x\\u(2,y) = 1\\u(0,y) = 1 \text{ for } 2 \leq y \leq 4\\u(1,y) = 0 \text{ for } 0 \leq y \leq 2\\u(x,2) = 0 \text{ for } 0 \leq x \leq 1\\u(x,0) = 1 \text{ for } 1 \leq x \leq 2$}
Using the central difference approximation for the second derivatives gives:
\begin{equation}
    \left(\frac{u_{i+1,j}^n - 2u_{i,j}^n + u_{i-1,j}^n}{\Delta x^2} + \frac{u_{i,j+1}^n - 2u_{i,j}^n + u_{i,j-1}^n}{\Delta y^2}\right) = f(x,y)
\end{equation}
Assuming that $h=\Delta x=\Delta y$ gives:
\begin{equation}
    u_{i,j} = \frac{1}{4}\left(u_{i+1,j} + u_{i-1,j} + u_{i,j+1} + u_{i,j-1} - h^2f(x,y)\right)
\end{equation}
In order to solve this, use the method of iterative relaxation, where the matrix is initilized with zeros in all entries except the boundary conditions and iterations are performed until the change from step to step is within some specified tolerance. In some sense, this can be though of as iterating through the transient part of the 2D heat equation until the steady-state solution is achieved. Shown below is an image of the heat on the surface as determined by the iteration. To have a figure which can be manipulated in space, please run the main.m file and select the corresponding option.
\begin{figure}[H]
    \centering
    \includegraphics[width=0.8\textwidth]{poisson_example.jpg}
    \caption{Final Numerical Solution of Poisson's Equation}
\end{figure}


\section{Spacecraft Application}
\subsection*{$
\triangledown^{2}u =0\\
u(0,y)  =0\\
u(a,y)  =0\\
u(x,0)  =0\\
u(x,b)  =\beta x(a-x)$}


By separation of variables we obtain an analytic temperature profile of 

\[
u(x,y)=\sum_{n=1}^{\infty}A_{n}\sin(8nx)\sinh(8ny)
\]

where  
\[
A_{n}=\frac{625\left(4\sin\left(\frac{\pi n}{2}\right)^{2}-\pi n\sin\left(\pi n\right)\right)}{2n^{3}\pi\sinh\left(\frac{\pi n}{2}\right)}.
\].
\begin{figure}[H]
    \centering
    \includegraphics[width=0.8\textwidth]{AnalyticalSolution.png}
    \caption{Analytical Solution of Radiator Fin}
\end{figure}

We can see quite clearly how temperature barely drops off the further away we are from the heat source so as such it would make sense to reduce the rectangle just to a parabola around the main heat source. This would provide the most amount of heat dissapation without greatly increasing weight.
Utilizing our formula from question 4, we can find that our answer seems to drop off quicker than the analytical solution. This is likewise also seen even more exaggerated in our MOL solution (which utilizes a very similar formula found in question 1) as seen in the figures below.
\begin{figure}[H]
    \centering 
    \includegraphics[width=0.6\textwidth]{FinPoisson.png}
    \caption{Poisson solution method of Radiator Fin}
\end{figure}
\begin{figure}[H]
    \centering
    \includegraphics[width=0.6\textwidth]{FinMOL.png}
    \caption{MOL solution method of Radiator Fin}
\end{figure}

While these methods generally followed the structure of the analytical solution, its especially clear in the MOL solution that it trended to 0 too quickly and as such didn't model the solution as good as it could have.


\includepdf[pages=-]{main.pdf}
\includepdf[pages=-]{basic_heat.pdf}
\includepdf[pages=-]{general_wave.pdf}
\includepdf[pages=-]{general_heat_2D.pdf}
\includepdf[pages=-]{general_poisson.pdf}


\end{document}